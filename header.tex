\documentclass{article}
\usepackage[utf8]{inputenc}
\usepackage{listings}
% \usepackage{xcolor}
\usepackage[table]{xcolor}
\usepackage{amsmath} 
\usepackage{cancel}
\usepackage{fancyhdr}
\usepackage{xfrac}
\usepackage{amssymb}
\usepackage{subfig}
\usepackage{graphicx}
\usepackage{wrapfig}
\usepackage{minted}
\usepackage{tikz}
\usepackage{pgfplots}
\usepackage{hyperref}
% fissa le footnotes in fondo alla pagina
\usepackage[bottom]{footmisc}
\usetikzlibrary{decorations.pathreplacing,calligraphy}

% FIX MARGINI
\usepackage{geometry}
\geometry{
a4paper,
total={170mm, 257mm},
left=20mm,
top=20mm,
}

% COMANDI CUSTOM
% \abovemargin{1pt} imposta il margine sopra una equazione \[ \] a 1 pt
\newcommand{\abovemargin}[1]{\setlength\abovedisplayskip{#1}}
% \s = \sigma
\newcommand{\s}{\sigma}
% \sumi = {\sum}_i
\newcommand{\sumi}{{\sum}_i}

% \uloosdot{parola} = parola sottolineata con lineette (utile per le footnotes)
\newcommand{\uloosdash}[1]{%
    \tikz[baseline=(todotted.base)]{
        \node[inner sep=1pt,outer sep=0pt] (todotted) {#1};
        \draw[loosely dashed] (todotted.south west) -- (todotted.south east);
    }%
}%

% DEFINIZIONE DEGLI SNIPPET DI CODICE
\definecolor{codegreen}{rgb}{0,0.6,0}
\definecolor{codegray}{rgb}{0.5,0.5,0.5}
\definecolor{codepurple}{rgb}{0.58,0,0.82}
\definecolor{backcolour}{rgb}{0.95,0.95,0.92}
\lstdefinestyle{style}{
    backgroundcolor=\color{backcolour},   
    commentstyle=\color{codegreen},
    keywordstyle=\color{magenta},
    numberstyle=\tiny\color{codegray},
    stringstyle=\color{codepurple},
    basicstyle=\ttfamily\footnotesize,
    breakatwhitespace=false,         
    breaklines=true,                 
    captionpos=b,                    
    keepspaces=true,                 
    numbers=left,                    
    numbersep=5pt,                  
    showspaces=false,                
    showstringspaces=false,
    showtabs=false,                  
    tabsize=2
}
\lstset{style=style}

% DEFINIZIONE COLORI
\definecolor{lightgreen}{rgb}{0.5,1,0.5}
\definecolor{lightred}{rgb}{1,0.5,0.5}
\definecolor{Orange}{HTML}{F58137}
\definecolor{ForestGreen}{HTML}{009B55}

\renewcommand{\figurename}{Figura}
\newcommand{\red}[1]{\color{red}{#1}\normalcolor}
\newcommand{\orange}[1]{\color{Orange}{#1}\normalcolor}
\newcommand{\green}[1]{\color{ForestGreen}{#1}\normalcolor}